
\startcomponent senv-slides
\environment slides-env

\setupTitle
  [ title={ПО сетевых устройств},
   author={Трещановский Павел Александрович, к.т.н.},
     date={\date},
  ]
\placeTitle

\SlideTitle {Стандартная C библиотека (libc.so)}
\startitemize
\item Взаимодействие с ОС прямо или косвенно осуществляется через libc.
\item Что входит в libc: библиотека языка C (malloc, free, strcmp), библиотека
для Unix (стандарт POSIX) (open, write, read), поддержка многопоточности
(pthread), математические функции и др.
\item Заголовочные файлы stdio.h, stdlib.h, string.h и многие другие являются
частью стандартной библиотеки.
\item libc.so выполняет первичную инициализацию процесса, а также завершение
процесса. Первая исполняемая функция - {\tt _start}. {\tt _start} в итоге
вызывает {\tt main}.
\item Существуют разные реализации: glibc (на ПК), musl (на встроенных
системах) и др.
\stopitemize

\SlideTitle {Системные вызовы}
{\tt printf} = формирование строки в буфере и вывод этой строки через функцию
{\tt write}. Как реализована {\tt write}?
\startitemize
\item Функции и код ядра Linux защищены аппаратно и не доступны приложениям.
\item Системные вызовы - программные прерывания, т.е. имеют аппаратную
поддержку в процессоре в виде специальных инструкций.
\item В целях безопасности ядро строго контролирует передаваемые аргументы.
\item Системные вызовы реализуются с помощью кода на ассемблере. Приложения
вызывают тонкие обертки из libc.
\item Основное взаимодействие между приложением и ядром осуществляется через
системные вызовы (open, close, read, write, brk и др.). 
\stopitemize

\SlideTitle {API для работы с файлами в ОС Linux}
\startitemize
\item
\starttyping
int open(const char *pathname, int flags);
\stoptyping
Функция {\tt open} возвращает файловый дескриптор. Флаги задают режим доступа: 
\startitemize
\item O_RDONLY - только чтение,
\item O_WRONLY - только запись,
\item O_RDWR - чтение и запись,
\item O_TRUNC - удаление всех данных из файла.
\stopitemize
\item
\starttyping
ssize_t read(int fildes, void *buf, size_t nbyte);
ssize_t write(int fildes, const void *buf, size_t nbyte);
\stoptyping
Чтение в буфер {\tt buf} или запись из буфера {\tt buf}. Количество - {\tt
nbyte} байтов.
\stopitemize

\SlideTitle {Что такое файл и файловый дескриптор}
\IncludePicture[horizontal][diagrams/unix-files.pdf]

\SlideTitle {Замечания}
\startitemize
\item У каждого процесса свое пространство файловых дескрипторов, совпадение
номеров не означает совпадение файлов.
\item API работает с дескриптором, а не с файлом, т.е. файл может быть
виртуальным. Пример полиморфизма.
\item Зачем нужны указатели (с дескрипторами) на открытия? Чтобы иметь
несколько указателей (в том числе из разных процессов) на одно и то же
открытие.
\item Дескриптор 0 (соответствует потоку stdin) - стандартный поток ввода, 1
(stdout) - поток вывода, 2 (stderr) - поток сообщений об ошибках.
\item {\tt printf} записывает сообщение в дескриптор 1. printf(...) = fprintf(stdout, ...).
\item Имена и пути к файлам - отдельный вопрос. Файл может существовать даже
без пути к нему.
\stopitemize

\SlideTitle {Потоки ввода-вывода}
\startitemize
\item
\starttyping
FILE *fopen(const char *path, const char *mode);
\stoptyping
Открытие файла. {\tt mode} - режим доступа, {\tt "r"} - чтение, {\tt "w"} -
запись с очисткой, {\tt "r+"} и {\tt "w+"} - чтение и запись.
\item
\starttyping
size_t fread(void *ptr, size_t size, size_t nmemb, FILE *stream);
size_t fwrite(const void *ptr, size_t size, size_t nmemb, FILE *stream);
\stoptyping
Чтение в буфер {\tt ptr} или запись из буфера {\tt ptr}. Объем данных в байтах
= размер блока {\tt size}, умноженный на количество блоков {\tt nmemb}.
\item Чтение и запись буферизируются.
\item Поток ввода-вывода содержит файловый дескриптор. На низком уровне чтение
и запись производятся через {\tt read} и {\tt write}.
\stopitemize

\SlideTitle {Создание нового процесса}
Запуск приложения = создание нового процесса + исполнение приложения в этом
процессе

\starttyping
pid_t fork(void);
\stoptyping

Что делает {\tt fork}.
\startitemize
\item Создает новый процесс, который исполняет то же приложение с момента
вызова {\tt fork}.
\item Для нового процесса выделяется идентификатор (PID, см. вывод команды {\tt
ps}).
\item Новый процесс является дочерним, тот, который вызвал {\tt fork} -
родительским (см. вывод команды {\tt pstree}).
\item Дочерний процесс наследует от родительского открытые файлы и
идентификатор пользователя.
\stopitemize

\SlideTitle {Исполнение приложения}
\startitemize
\item
\starttyping
int execve(const char *filename, char *const argv[], char *const envp[]);
\stoptyping
Функция заменяет текущее приложение на новое и передает аргументы командной
строки.
\item
\starttyping
void exit(int status);
\stoptyping
Завершение процесса с заданным кодом. Посмотреть код из shell: {\tt echo \$?}
\item Функция {\tt waitpid} позволяет дождаться завершения дочернего процесса с
заданным PID.
\item
\starttyping
int system(const char *command);
\stoptyping
Функция создает новый процесс, запускает в нем {\tt /bin/sh} с командой {\tt
command} и дожидается его завершения.
\stopitemize

\SlideTitle {Аргументы командной строки}
{\tt \$ ls {\color[darkred] \en-\en-width 50 -l} {\color[blue] /home/student/} }

\startitemize
\item Позиционные аргументы (синий) - порядок следования определяет смысл.
\item Именованные аргументы (красные) - имя аргумента задается явно в
короткой (-w) или длинной (\en-\en-width) форме.
\item Обычно именованные аргументы являются необязательными.
\item У именованного аргумента может быть значение.
\item С точки зрения ОС все аргументы - это массив строк. Классификацию
аргументов осуществляет приложение.
\stopitemize

\SlideTitle {Анализ именованных аргументов с помощью getopt}
\starttyping
	int opt;
	while ((opt = getopt(argc, argv, "cLd:")) != -1) {
		switch (opt) {
		case 'c':
			has_c = true; break;
		case 'd':
			debug_level = atoi(optarg); break;
		...
	}
\stoptyping
\startitemize
\item Поддерживается только короткая форма именованных аргументов.
\item getopt вызывается в цикле. На каждой итерации анализируется один аргумент.
\item Поддерживаются аргументы со значением (: в дескрипторе). Строка со
значением доступна через указатель {\tt optarg}.
\item Меняет порядок аргументов (только в glibc!!!). Все позиционные аргументы
переносятся в конец {\tt argv}.
\stopitemize

\SlideTitle {Сетевой стек Linux}
\IncludePicture[horizontal][diagrams/network-stack.pdf]

\SlideTitle {Сетевые интерфейсы}
\starttyping
$ ip addr
...
enp3s0: <BROADCAST,MULTICAST,UP,LOWER_UP> mtu 1500 master lxcbr1 state UP
    link/ether 08:60:6e:81:12:ac brd ff:ff:ff:ff:ff:ff
    inet 172.17.17.105/24 brd 172.17.17.255 scope global enp3s0
\stoptyping
\startitemize
\item Флаг NO-CARRIER - отсутствие физического (\quote{несущего}) сигнала.
\item Флаг UP -- административное включение интерфейса.
\item Состояние UP/DOWN - наличие/отсутствие соединения.
\item MTU - maximum transmission unit, максимальный размер пакета, который
можно передать без фрагментации.
\item IP-адрес (172.17.17.105/24) - добавляется в таблицу маршрутизации (вывод
по команде {\tt ip route}). Маска (/24) определяет множество адресов, которые
напрямую достижимы через данный интерфейс.
\item MAC-адрес - адрес интерфейса в локальной сети Ethernet.
\stopitemize

\SlideTitle {Настройка интерфейсов через iproute2}
\startitemize
\item Удаление всех IP-адресов с интерфейса {\tt ge1}:
\starttyping
# ip addr flush dev ge1
\stoptyping
\item Добавление адреса 192.168.0.9 с маской /24 на интерфейс {\tt ge1}:
\starttyping
# ip addr add dev ge1 192.168.0.9/24
\stoptyping
\item Выключение интерфейса {\tt ge1}:
\starttyping
# ip link set dev ge1 down
\stoptyping
\item Включение интерфейса {\tt ge1}:
\starttyping
# ip link set dev ge1 up
\stoptyping
\item Установка MAC-адреса на {\tt ge1}:
\starttyping
# ip link set dev ge1 address 22:22:22:22:22:22
\stoptyping
\stopitemize

Настройки не сохраняются после перезагрузки!

\stopcomponent
