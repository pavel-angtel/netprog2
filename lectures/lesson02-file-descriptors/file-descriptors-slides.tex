
\startcomponent file-descriptors-slides
\environment slides-env

\setupTitle
  [ title={ПО сетевых устройств},
   author={Трещановский Павел Александрович, к.т.н.},
     date={\date},
  ]
\placeTitle

\SlideTitle {Multics vs Unix}
Multics
\startitemize
\item Совместная разработка MIT, Bell Labs и General Electric.
\item Высокоуровневый язык программирования PL/1.
\item Инновационные идеи: файловая система, виртуальная страничная память,
динамические библиотеки и др.
\item Через пять лет разработки (64-69) - отсутствие рабочего продукта.
\stopitemize

Unix
\startitemize
\item 4-5 разработчиков первоначальной версии.
\item Жесткие аппаратные ограничения.
\item Минимальный функционал.
\item Рабочий прототип был получен после нескольких месяцев разработки в 1969
году.
\stopitemize

\SlideTitle {Top-Down vs Bottom-Up}
\IncludePicture[horizontal][diagrams/td-bu.pdf]

\SlideTitle {Преимущества и недостатки}
\startitemize
\item Bottom-Up позволяет быстрее получить рабочий прототип.
\item Bottom-Up лучше приспособлен к эволюции технических требований и
возможностей. Части системы можно менять и/или адаптировать к новым условиям.
\item Top-Down обычно более полно удовлетворяет исходным требованиям.
\item Примеры Bottom-Up решений - файлы, файловые системы, большинство
операционных систем.
\stopitemize

\SlideTitle {Принципы разработки ОС Unix}
McIlroy, предисловие к Unix Time-Sharing System (1978):
\startitemize
\item Make each program do one thing well. To do a new job, build afresh rather
than complicate old programs by adding new \quote{features}.
\item Expect the output of every program to become the input to another, as yet
unknown, program. Don't clutter output with extraneous information. Avoid
stringently columnar or binary input formats. Don't insist on interactive
input.
\item Design and build software, even operating systems, to be tried early,
ideally within weeks. Don't hesitate to throw away the clumsy parts and rebuild
them.
\stopitemize

\SlideTitle {Иерархия каталогов и файлов}
\IncludePicture[horizontal][diagrams/file-hierarchy.pdf]

\SlideTitle {Иерархия каталогов, замечания}
\startitemize
\item У каждого процесса есть текущий каталог.
\item Абсолютный путь - начинается с {\tt /} и перечисляет имена каталогов от
корня до цели. Например, {\tt /home/root/share/netprog2}.
\item Относительный путь - перечисляет имена каталогов от текущего каталога до
цели. Например, {\tt share/netprog2} или {\tt ./share/netprog2}.
\item {\tt \sim} (тильда) эквивалентна пути к домашнему каталогу пользователя.
Для пользователя root - {\tt /home/root}.
\stopitemize

\SlideTitle {Внутреннее представление в ядре ОС}
\IncludePicture[horizontal][diagrams/inode-dentry.pdf]

\SlideTitle {Метаданные}
\startitemize
\item Владелец файла.
\item Тип файла (обычный, каталог, символьный, блочный).
\item Права доступа к файлу - rwx, где r - доступ на чтение, w - на запись, x -
на исполнение.
\item Права доступа независимо задаются для владельца файла, для группы
пользователей, к которой принадлежит владелец, и для всех остальных
пользователей.
\item Права часто записываются в виде 3 цифр. Например, 755 (111 101 101 в
двоичной записи) разрешает любой доступ для владельца, но только чтение и
исполнение для его группы и прочих пользователей.
\item Время создания и последнего изменения файла.
\item Размер файла.
\stopitemize

\SlideTitle {Просмотр и изменение метаданных}

\startitemize
\item Просмотр метаданных:
\starttyping
$ ls -l
drwxr-xr-x 18 pavel pavel 4096 27 févr. 22:35 lab-assignments
\stoptyping
d - каталог, 18 - количество файлов внутри.
\item Задание прав доступа:
\starttyping
$ chmod 777 ./Makefile
\stoptyping
\item Задание владельца:
\starttyping
$ chown pavel:pavel ./Makefile
\stoptyping
\item Обновление времени доступа к файлу:
\starttyping
$ touch lab_00.c
\stoptyping
\stopitemize

\SlideTitle {Специальные файлы}

\startitemize
\item Доступ к большинству периферийных устройств осуществляется через
специальные файлы - символьные и блочные файлы.
\item Символьные файлы предоставляют доступ к устройствам, передающим и
принимающим поток байтов. Например, последовательный интерфейс, модем GSM,
терминал.
\item Блочные файлы предоставляют доступ к устройствам, предоставляющим
произвольный доступ к массиву байтов. Например, диски и разделы.
\item Метаданные специальных файлов содержат 2 числа - старший и младший
номера. Эти числа идентифицируют устройство. Имя файла не имеет значения для
ядра ОС.
\item Обычно файлы устройств находятся в каталоге /dev: /dev/tty1 - терминал,
/dev/sda - жесткий диск и т.д.
\stopitemize

\SlideTitle {Специальные файлы, замечания}

\startitemize
\item Сами файлы не содержат никаких данных - все операции чтения и записи
реализуются нижележащими устройствами.
\item Файл - это не само устройство, а только интерфейс к нему. Наличие или
отсутствие файла не означает наличие или отсутствие устройства.
\item Обычно создаются и удаляются системой автоматически.
\item Могут быть созданы вручную:
\starttyping
$ mknod another_tty1 c 4 1
\stoptyping
Создает символьный файл для терминала (старший номер 4, младший номер 1).
\stopitemize

\SlideTitle {Диски и разделы}
\IncludePicture[horizontal][diagrams/disks-volumes.pdf]

\SlideTitle {Точки монтирования}
\IncludePicture[horizontal][diagrams/mounting-points.pdf]

\SlideTitle {Командный интерфейс}
\startitemize
\item Монтирование:
\starttyping
$ mount -t ext4 /dev/sda1 /mnt
\stoptyping
\item Аргументы: тип (ext4), что (раздел /dev/sda1), куда (/mnt).
\item Демонтирование:
\starttyping
$ umount /mnt
\stoptyping
\item Отображение точек монтирования:
\starttyping
$ mount
/dev/sda1 on / type ext4 (rw,noatime,data=ordered)
proc on /proc type proc (rw,nosuid,nodev,noexec,relatime)
devpts on /dev/pts type devpts (rw,relatime,gid=5,mode=620,ptmxmode=000)
sysfs on /sys type sysfs (rw,nosuid,nodev,noexec,relatime)
\stoptyping
\stopitemize

\SlideTitle {Преимущества монтируемых файловых систем}
\startitemize
\item Косметика: отображение файловых систем на жестких дисках в произвольных
точках дерева файлов по вкусу администратора.
\item Более глубокое преимущество - отвязка дерева файлов от физического
носителя данных, \quote{виртуализация} файловой системы.
\item Файловая система может предоставлять доступ внутренним объектам ядра
Linux в целях конфигурирования, мониторинга и диагностики.
\item Файловая система может предоставлять доступ к внешним ресурсам:
FTP-сервер, проект в системе контроля версий git, база данных SQL.
\item Для приложений виртуальная ФС выглядит как обычное (почти) дерево файлов.
\stopitemize

\SlideTitle {Виртуальные ФС proc и tmpfs}
\startitemize
\item proc предоставляет доступ к информации о процессах и общей информации о системе.
\item Многие утилиты используют proc. Например, утилита ps:
\starttyping
$ ps -A
PID TTY          TIME CMD
  1 ?        00:00:03 init
  2 ?        00:00:00 kthreadd
  3 ?        00:00:00 ksoftirqd/0
\stoptyping
\item tmpfs представляет часть оперативной памяти в виде файловой системы.
\item tmpfs используется для временных файлов, которые не должны сохранятся при
перезапуске системы, а также в случаях, когда нужен очень быстрый обмен данными
через файлы.
\stopitemize

\SlideTitle {Открытый файл и файловый дескриптор}
\IncludePicture[horizontal][diagrams/open-files.pdf]

\SlideTitle {Низкоуровневый API}
\starttyping
int fd;
char string[] = "sdfgsdfg";
int ret;

/* int open(const char *pathname, int flags); */
fd = open("/dir1/dir2/file1", O_WRONLY | O_TRUNC);
if (fd < 0) {
	printf("open failed: %d\n", strerror(errno));
	goto on_error;
}

/* ssize_t write(int fd, const void *buf, size_t count); */
ret = write(fd, write_buffer, strlen(string));
if (ret < 0) {
	printf("write failed: %d\n", strerror(errno));
	goto on_error;
}

close(fd);
\stoptyping

\SlideTitle {Дескрипторы, замечания}
\startitemize
\item API: open(), close(), write(), read().
\item Флаги {\tt O_RDONLY}, {\tt O_WRONLY}, {\tt O_RDWR} открывают на чтение,
на запись, на чтение и запись.
\item Флаг {\tt O_TRUNC} удаляет имеющееся содержимое файла.
\item Флаги можно объединять операцией побитового или {\tt \|}.
\item Обычно у процесса при запуске уже созданы дескрипторы 0, 1 и 2.
\stopitemize

\SlideTitle {Потоки ввода-вывода}
\IncludePicture[horizontal][diagrams/io-streams.pdf]

\SlideTitle {Более высокоуровневый API}
\starttyping
FILE *fp;
char string1[] = "sdfgsdfg";
char string2[] = "32342";

/* FILE *fopen (const char *path, const char *mode); */
fp = fopen("/dir1/dir2/file1", "w");
if (!fp) {
	printf("fopen failed: %d\n", strerror(errno));
	goto on_error;
}

/* size_t fwrite (const void *ptr, size_t size, size_t nmemb, FILE *stream); */
fwrite(string1, 1, strlen(string1), fp);
fwrite(string2, 1, strlen(string2), fp);

fclose(fp);
\stoptyping

\SlideTitle {Исключительные ситуации}
\starttyping
FILE *fp_in, *fp_out;
char buf[128];
size_t sz;

/*
 * Открытие файлов.
 *  .......
 */

/* size_t fread (void *ptr, size_t size, size_t nmemb, FILE *stream); */
while (1) {
	sz = fread(buf, 1, sizeof(buf), fp_in);
	fwrite(buf, 1, sz, fp_out);

	/* fread()/fwrite() _не_ возвращают -1 в случае ошибки. */
	if (ferror(fp_in) || ferror(fp_out))
		goto on_error;
	if (feof(fp_in))
		break;
}
\stoptyping

\SlideTitle {Буферизация}
\IncludePicture[horizontal][diagrams/io-buffering.pdf]

\SlideTitle {Потоки, замечания}
\startitemize
\item Буферизация данных на чтение и запись.
\item Более удобный API.
\item {\tt fprintf()} - форматированный вывод в указанный поток.
\item {\tt fgets()} - чтение строки (последовательность, оканчивающаяся на
'\textbackslash n') из указанного потока.
\item Каждый поток однозначно соответствует дескриптору. {\tt fdopen()} создает
поток из дескриптора.
\item Стандартные потоки ввода (stdin), вывода (stdout) и ошибок создаются из
дескрипторов 0, 1 и 2 при запуске процесса.
\stopitemize

\SlideTitle {Конвейер (pipe)}
\IncludePicture[horizontal][diagrams/pipe.pdf]

\SlideTitle {Использование pipe из shell}
Команда
\starttyping
prog1 | prog2
\stoptyping
подключает дескриптор 1 (вывод) программы prog1 к дескриптору 0 (ввод)
программы prog2 через конвейер. prog1 и prog2 не знают о существовании
конвейера.

Примеры:

{\tt 'cat students.txt \| grep Маша'} выводит только строки с указанным именем
в терминал.

{\tt 'cat students.txt \| sed s/Маша/Даша/'} производит замену имени и выводит
результат в терминал.

{\tt 'cat students.txt \| wc -l'}  подсчитывает количество строк в файле.

\\

Конвейер может иметь произвольную длину:
\starttyping
prog1 | prog2 | prog3 | prog4 | ...
\stoptyping

\SlideTitle {Управление потоками ввода-вывода}
\IncludePicture[horizontal][diagrams/data-redirection.pdf]

\SlideTitle {Перенаправление в shell}
Команда
\starttyping
prog1 < file1 > file2
\stoptyping
подключает дескриптор 0 (ввод) программы prog1 к файлу file1, а дескриптор 1
(вывод) - к файлу file2. prog1 не знает о перенаправлении.

{\tt 'grep Маша < students.txt > output.txt'} ищет строки с указанным именем в
файле students.txt и записывает их в файл output.txt.

\SlideTitle {Применение принципов разработки ОС Unix}
\startitemize
\item Программы принимают на вход и передают на выход текстовые потоки.
\item Система предоставляет набор инструментов (утилит) для решения
элементарных задач.
\item Сложные задачи решаются связыванием нескольких утилит через файлы и
конвейеры.
\item Система предоставляет утилиты-фильтры для обработки потоков между
приложениями: поиск (grep), замена(sed) и т.д.
\item shell является универсальным клеем. Отдельные команды объединяются в
скрипты.
\stopitemize

\SlideTitle {Шаблоны поиска (glob pattern)}
Пример:
\starttyping
$ ls *.pd?
data-redirection.pdf  file-hierarchy.pdf
\stoptyping
\startitemize
\item shell автоматически подменяет шаблон на последовательность
соответствующих имен.
\item Специальному символу {\tt *} соответствует любая последовательность
обычных символов (в т.ч. последовательность длины 0).
\item Специальному символу {\tt ?} соответствует любой один обычный символ.
\item Специальные символы могут использоваться в одном шаблоне произвольное
число раз.
\stopitemize

\stopcomponent
