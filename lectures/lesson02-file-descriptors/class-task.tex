
\startcomponent class-task
\environment practice-env

\subsection{Упражнения к теме \quote{Файлы и файловые дескрипторы}}

\startitemize[n]
\item Выведите текущий каталог:
\starttyping
$ pwd
\stoptyping
\item Перейдите в каталог {\tt /usr/lib} с использованием абсолютного пути:
\starttyping
$ cd /usr/lib
\stoptyping
\item Перейдите в каталог {\tt /home/root} с использованием относительного пути:
\starttyping
$ cd ../../home/root
\stoptyping
\item Создайте файл {\tt file1}:
\starttyping
$ touch file1
\stoptyping
\item Выведите метаданные этого файла:
\starttyping
$ ls -l file1
\stoptyping
\item Выведите метаданные с точным временем последнего изменения:
\starttyping
$ ls --full-time file1
\stoptyping
\item Повторите команду {\tt touch file1}. Что изменилось в метаданных?
\item Попробуйте запустить файл {\tt file1}:
\starttyping
$ ./file1
\stoptyping
Почему ядро отказалось исполнять его?
\item Разрешите исполнение файла для "остальных" пользователей:
\starttyping
$ chmod 645 ./file1
\stoptyping
Можно ли его исполнить теперь?
\item Разрешите исполнение файла для владельца файла. Убедитесь в том, что
теперь файл исполняется.
\item Выведите метаданные для файлов {\tt /dev/tty1} и {\tt /dev/sda1}. Какой
файл является символьным, а какой блочным? Какие старшие и младшие номера
устройств?
\item Удалите файл:
\starttyping
$ rm ./file1
\stoptyping
\item Выведите текст в терминал с помощью команды {\tt echo}:
\starttyping
$ echo 123
\stoptyping
\item Выведите текст в терминал с использованием управляющих
последовательностей \backslash b и \backslash n:
\starttyping
$ echo -e aaa\\nbbb
$ echo -e aaa\\bccc
\stoptyping
Замечание. shell удаляет обратную косую черту из аргументов команды. Чтобы
передать одну косую черту в команду echo, ее необходимо ввести дважды. В
домашнем задании двойную косую черту использовать не нужно.
\item Запишите произвольный текст в файл students с помощью перенаправления
вывода:
\starttyping
$ echo aaa > students
\stoptyping
\item Убедитесь в том, что каждая последующая команда {\tt echo} перезаписывает
содержимое файла.
\item Использую управляющую последовательность \backslash n, создайте файл {\tt
students}, содержащий строки Masha, Sasha, Vika.
\item Выведите содержимое файла:
\starttyping
$ cat ./students
\stoptyping
\item Подключите к выводу {\tt cat} программу поиска текста {\tt grep} с помощью
конвейера:
\starttyping
$ cat ./students | grep asha
\stoptyping
\item Подключите к выводу {\tt grep} потоковый текстовый редактор {\tt sed} и
замените слово Masha на Dasha:
\starttyping
$ cat ./students | sed s/Masha/Dasha/g
\stoptyping
\item Используя утилиты-фильтры {\tt grep} и {\tt sed}, создайте на основе
students новый файл students2, содержащий только имена с суффиксом \quote{asha}
и в котором маленькие буквы \quote{a} заменены на большие \quote{A}.
\item Удалите созданные файлы:
\starttyping
$ rm ./students
$ rm ./students2
\stoptyping
\item Перейдите в каталог {\tt /dev}. Далее использую shell-шаблоны, выведите список 
файлов с префиксом {\tt tty}:
\starttyping
$ ls tty*
\stoptyping
\item Объясните вывод следующих команд:
\starttyping
$ ls tty4*
$ ls *4*
$ ls tty49*
$ ls tty49?
$ ls tty4?
\stoptyping

\stopitemize

\stopcomponent
