
\startcomponent program-structure-slides
\environment slides-env

\setupTitle
  [ title={ПО сетевых устройств},
   author={Трещановский Павел Александрович, к.т.н.},
     date={\date},
  ]
\placeTitle

\SlideTitle {Текущее положение}
\startitemize
\item Сервер, принимающий и обрабатывающий TCP-соединения с помощью сокета.
\item Внутренние объекты сервера могут представляться в виде набора сетевых
ресурсов. Чтение и изменение ресурсов производится через вызов методов (GET,
POST, PUT и т.д.).
\item Сервер поддерживает одновременное исполнение нескольких задач с помощью
цикла обработки событий.
\item Задача может представлять собой исполнение экземпляра конечного автомата.
\stopitemize

Какие еще бывают задачи? Какого их внутренняя структура? Как организовано
взаимодействие между их компонентами?

\SlideTitle {Отдельные аспекты программной архитектуры}
\startitemize
\item Разбиение на уровни абстракции.
\item Динамическое создание и удаление ресурсов.
\item Полиморфные ресурсы - общий API для ресурсов с разной реализацией.
\item Эффективное индексирование ресурсов с помощью словарей.
\item Обмен данными между уровнями абстракции через очередь событий.
\stopitemize

\SlideTitle {Многоуровневое приложение}
\IncludePicture[horizontal][diagrams/layering.pdf]

\SlideTitle {Многоуровневое приложение, код}
\starttyping
int post(struct resource *res, const char *http_request)
{
	struct implementation *impl = res->implementation;

	parse_http_request(http_request, ...);

	impl_post(impl, ...);

	send_http_response(...);
}

int impl_post(struct implementation *impl, ...)
{
	run_state_machine(impl->state, event);
}

/* Функция скрыта от высокоуровневого модуля */
static run_state_machine(enum state state, enum event event)
{
	...
}
\stoptyping

\SlideTitle {Динамическое создание ресурсов}
\IncludePicture[horizontal][diagrams/dynamic-resources.pdf]

\SlideTitle {Создание и регистрация ресурсов}
\starttyping
int register_resource(struct implementation *impl, ...)
{
	struct resource *res;

	res = calloc(1, sizeof(*res));
	res->implementation = impl;
}

int put(const char *http_request)
{
	struct resource *res;
	struct implementation *impl;

	...

	res = calloc(1, sizeof(*res));

	impl = create_implementation(res, ...);
	res->implementation = impl;
}
\stoptyping


\SlideTitle {Поддержка нескольких реализаций}
\IncludePicture[horizontal][diagrams/multiple-implementations.pdf]

\SlideTitle {Простой вариант}
\starttyping
int register_resource(enum resource_type type, struct implementation1 *impl1, ...)
{
	struct resource *res = calloc(1, sizeof(*res));

	res->type = type;
	if (type == IMPL1)
		res->implementation1 = impl1;
}

int post(struct resource *res, const char *http_request)
{
	...

	switch (res->type) {
	case IMPL1:
		impl1_post(impl, ...);
		break;
	case IMPL2:
		impl2_post(impl, ...);
		break;
	}
}
\stoptyping

\SlideTitle {Обратный вызов}
\starttyping
typedef int (*callback_t)(void *callback_data);

int register_resource(callback_t callback, void *callback_data)
{
	struct resource *res = calloc(1, sizeof(*res));

	res->callback = callback;
	res->callback_data = callback_data;
}

int post(struct resource *res, const char *http_request)
{
	....

	res->callback(res->callback_data);
}

static int impl1_post(void *callback_data)
{
	struct implementation1 *impl = callback_data;
	...
}
\stoptyping

\SlideTitle {Замечания}
\startitemize
\item Обратный вызов - функция, которая вызывается через указатель.
\item {\tt callback_t} - тип этой функции.
\item {\tt callback_data} - данные, которые надо передать в обратный вызов во
время его исполнения. Тип {\tt void *} позволяет каждой реализации ресурса
использовать данные своего типа (например, {\tt struct implementation1} или
{\tt struct implementation2}). Приведение от указателя на структуру к {\tt void
*} и наоборот происходит автоматически.
\item Указатель на функцию и указатель {\tt callback_data} записывается в
объект {\tt struct resource} во время регистрации ресурса ({\tt
resource_register())}.
\stopitemize

\SlideTitle {Таблица функций}
\starttyping
struct resource_operations {
	int (*get)(struct resource *res, ...);
	int (*post)(struct resource *res, ...);
};

int register_resource(struct resource_operations *ops, void *private_data)
{
	res->operations = ops;
	res->private_data = private_data;
}

int post(struct resource *res, const char *http_request)
{
	....
	res->operations->post(res);
}

static int impl1_post(struct resource *res)
{
	struct implementation1 *impl = res->private_data;
	...
}
\stoptyping

\SlideTitle {Индексация ресурсов}
\IncludePicture[horizontal][diagrams/resource-indexing.pdf]

\SlideTitle {Обычный массив}
\externalfigure[diagrams/simple-dictionary.pdf][height=0.16\textheight]
\starttyping
struct resource {
	char *key;
	struct implementation *impl;
} *resource[MAX_RESOURCES];

struct resource *find_resource(const char *key)
{
	int i;

	for (i = 0; i < MAX_RESOURCE; i++) {
		struct resource *r = &resources[i];

		if (r && !strcmp(r->key, key))
			return r;
	}
	return NULL;
}
\stoptyping

\SlideTitle {Двоичное дерево}
\IncludePicture[horizontal][diagrams/binary-tree-dictionary.pdf]

\SlideTitle {Балансирование дерева}
\IncludePicture[horizontal][diagrams/tree-balancing.pdf]

\SlideTitle {Самобалансирующееся дерево}
\startitemize
\item Время поиска элемента словаря в худшем случае определяется глубиной
дерева. Наименьшее время достигается при равномерном распределении узлов по
всем ветвям дерева.
\item Самобалансирующееся дерево - дерево, у которого распределение узлов по
ветвям происходит автоматически при добавлении нового узла.
\item AVL - одно из самобалансирующихся деревьев.
\item Библиотека libubox предоставляет реализацию деревьев AVL - {\tt struct
avl_tree}.
\item {\tt struct avl_tree} скрывает двоичное дерево и предоставляет приложение
API ассоциативного массива (словаря) с возможностью добавления и поиска по ключу.
\item Ключом может быть объект произвольного типа, в частности, строка. При
инициализации словарю надо передавать функцию сравнения ключей данного типа
({\tt avl_strcmp} для строк).
\stopitemize

\SlideTitle {struct avl_tree}
\IncludePicture[horizontal][diagrams/avl-tree.pdf]

\SlideTitle {Преобразование указателей}
\starttyping
struct resource {
	struct avl_node anode;
};
\stoptyping

Мы добавляем и извлекаем из словаря объекты struct avl_node, встроенные в
struct resource.  Как из указателя на struct avl_node получить указатель на
struct resource?

\starttyping
/* Смещение между адресом родительского объекта типа type и адресом его поля
member. */
#define offsetof(type, member)	((size_t)&((type *)0)->member)

/* Получение указателя на родительский объект типа type из указателя ptr на его
поле member. */
#define container_of(ptr, type, member) ({		\
	void *__mptr = (void *)(ptr);			\
	((type *)(__mptr - offsetof(type, member))); })

int function(struct avl_node *n)
{
	struct resource *res = container_of(n, struct resource, anode);
}
\stoptyping


\SlideTitle {Добавление, поиск и удаление}
\starttyping
struct avl_tree dict;
struct resource {
	char *key;
	struct avl_node anode;
} *res1, *res2;

/* avl_strcmp - функция сравнения узлов двоичного дерева. */
avl_init(&dict, avl_strcmp, 0, NULL);

res1 = calloc(1, sizeof(*res1));
res1->key = strdup("some-resource-key");
res1->anode.key = res1->key;

ret = avl_insert(&dict, &res1->anode);
if (ret < 0)
	fprintf(stderr, "Key '%s' is already present", res1->key);

res2 = avl_find_element(&dict, "another-key", res2, anode);

avl_delete(&dict, &res2->anode);
free(res2->key);
free(res2);
\stoptyping

\SlideTitle {Перебор всех элементов}
\starttyping
struct avl_tree dict;

struct resource *res, *next_res;

avl_init(&dict, avl_strcmp, 0, NULL);

...

avl_for_each_element(&dict, res, anode) {
	printf("%s\n", res->key);
	/* Нельзя удалять res из словаря!! */
}

/* Если в теле цикла удаляются элементы словаря, то используется вариант _safe. */
avl_for_each_element_safe(&dict, res, anode, next_res) {
	avl_delete(&dict, &res->anode);
	free(res->key);
	free(res);
}
\stoptyping

\SlideTitle {Сетевой ресурс с очередью событий}
\IncludePicture[horizontal][diagrams/event-queue.pdf]

\SlideTitle {Связный список struct list_head}
\IncludePicture[horizontal][diagrams/generic-list.pdf]

\SlideTitle {Добавление, поиск и удаление}
\starttyping
struct list_head list;
struct event {
	struct list_head entry;
} *event1, *event2;

INIT_LIST_HEAD(&list);

event1 = calloc(1, sizeof(*event1));

/* Добавление в конец списка. */
list_add_tail(&event1->entry, &list);

event2 = list_first_entry(&list, struct event, entry);

list_del(&event2->entry);
\stoptyping

\SlideTitle {Перебор всех элементов}
\starttyping
struct list_head list;

struct event *event, *next_event;
int i = 0;

list_for_each_entry(event, &list, entry) {
	printf("List entry %d\n", i++);
	/* Нельзя удалять event из списка!! */
}

/* Если в теле цикла удаляются элементы списка, то используется вариант _safe. */
list_for_each_entry_safe(event, next_event, &list, entry) {
	list_del(&event->entry);
	free(event);
}
\stoptyping

\SlideTitle {Замечания}
\startitemize
\item Объекты типа {\tt struct list_head} выполняют 2 роли: заголовок (список в
целом) и элемент списка. Элемент списка list_head должен быть встроен в {\tt
struct event} (или другую пользовательскую структуру). Заголовок {\em не}
включается в {\tt struct resource}.
\item Пустой список представляется в программе заголовком без элементов списка.
Заголовок в таком случае указывает сам на себя.
\item Каждое поле типа {\tt struct list_head} внутри {\tt struct resource}
отвечает за включение ресурса в один список. Если требуется включить ресурс в N
списков, то в структуре должно быть N полей {\tt struct list_head}.
\stopitemize

\SlideTitle {Устройства ввода в Linux}
\startitemize
\item Устройства ввода (клавиатура, мышь, кнопки на корпусе и пр.) представлены
в Linux в виде файлов {\em символьных устройства} в каталоге /dev/input.
\item Физическое устройство ввода соответствует одному или нескольким файлам в
/dev/input.
\item Каждое символьное устройство содержит очередь событий.
\item Событие имеет тип (нажатие клавиши, движение мыши), код (какая клавиша
нажата) и значение (нажата или отжата).
\item Приложения работают с устройствами через библиотеку libevdev.
\stopitemize

\SlideTitle {Опрос датчиков}
\IncludePicture[horizontal][diagrams/sensors.pdf]

\SlideTitle {Сканирование}
\starttyping
static int scan_devices(struct avl_tree *devices)
{
	int fd;
	struct libevdev *evdev;

	fd = open("/dev/input/event5", O_RDONLY | O_NONBLOCK);
	libevdev_new_from_fd(fd, &evdev);

	if (libevdev_has_event_type(evdev, EV_REL) &&
	    libevdev_has_event_type(evdev, EV_ABS)) {
		struct mouse *mouse;

		mouse = calloc(1, sizeof(*mouse));
		mouse->evdev = evdev;

		/* Добавление устройства в словарь,
		   регистрация обратного вызова read_mouse_event. */
		ret = register_device(devices, "mouse", evdev,
				      NULL, read_mouse_event, mouse);
	}
}
\stoptyping

\SlideTitle {Чтение событий evdev}
\starttyping
static int read_mouse_event(char value[], void *dev)
{
	struct mouse *mouse = dev;
	struct input_event ev;

	while (1) {
		ret = libevdev_next_event(mouse->evdev, LIBEVDEV_READ_FLAG_NORMAL, &ev);
		/* События может не быть в очереди. libevdev передает значение
		errno в коде статуса.*/
		if (ret == -EAGAIN)
			return 0;

		if (ev.type == EV_KEY && ev.code == BTN_LEFT && ev.value == 1) {
			/* Нажата левая кнопка мыши. */
			break;
		} else if (ev.type == EV_KEY && ev.code == BTN_RIGHT && ev.value == 1) {
			/* Нажата правая кнопка мыши. */
			break;
		}
	}
	return 0;
}
\stoptyping

\SlideTitle {Задание}
\IncludePicture[horizontal][diagrams/sensors-task.pdf]

\stopcomponent
