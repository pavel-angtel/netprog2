
\startcomponent class-task
\environment practice-env

\subsection{Упражнения к теме \quote{Клиентские сокеты}}

\startitemize[n]
\item В AngtelOS создайте файл, доступный встроенному FTP-серверу:
\starttyping
# echo 123 > /home/root/file1
\stoptyping
\item На физической машине запустите анализатор пакетов Wireshark. Включите
перехват пакетов на сетевом интерфейсе с адресом 192.168.3.6.
\item На физической машине запустите консольный FTP-клиент.
\item Установите соединение с сервером:
\starttyping
ftp> open 192.168.3.8
Connected to 192.168.3.8 (192.168.3.8).
220 Operation successful
Name (192.168.3.8:pavel): root
530 Login with USER+PASS
SSL not available
331 Specify password
Password:
230 Operation successful
Remote system type is UNIX.
Using binary mode to transfer files.
\stoptyping
\item Изучите процесс установки соединения в выводе Wireshark. Какие номера
портов TCP используют клиент и сервер? Какие пакеты обеспечивают установку
соединения TCP? Кто передает первый пакет FTP?
\item Введите команды {\tt system} и {\tt pwd}:
\starttyping
ftp> system
215 UNIX Type: L8
ftp> pwd
257 "/"
\stoptyping
Сопоставьте вывод клиента с содержимым переданных пакетов. Какие коды и
сообщения содержатся в ответах от сервера?
\item Включите пассивный режим на клиенте:
\starttyping
ftp> passive
Passive mode on.
\stoptyping
\item Выведите список файлов, находящихся в текущем каталоге сервера:
\starttyping
ftp> ls
227 PASV ok (192,168,3,8,140,229)
150 Directory listing
-rw-r--r--    1 0        0                4 Jan  1 00:40 file1
226 Operation successful
\stoptyping
Как сервер сообщает клиенту номер порта для дополнительного TCP-соединения? По
выводу Wireshark убедитесь, что клиент использует номер порта, указанный
сервером.
\item Скачайте с сервера файл {\tt file1}:
\starttyping
ftp> get file1
local: file1 remote: file1
227 PASV ok (192,168,3,8,144,23)
150 Opening BINARY connection for file1 (4 bytes)
226 Operation successful
4 bytes received in 0,000168 secs (23 Kbytes/sec)
\stoptyping
\item Закройте соединение с сервером:
\starttyping
ftp> quit
221 Operation successful
\stoptyping
\item Удалите файл file1:
\starttyping
# rm /home/root/file1
\stoptyping
\item Напишите программу {\tt ftp-client}, которая устанавливает соединение с
локальным FTP-сервером (адрес 127.0.0.1, порт 21), принимает первый пакет от
сервера, извлекает из него код и текст сообщения, после чего завершает
соединение с сервером отправкой команды QUIT.
\stopitemize

\stopcomponent
