
\startcomponent client-sockets-slides
\environment slides-env

\setupTitle
  [ title={ПО сетевых устройств},
   author={Трещановский Павел Александрович, к.т.н.},
     date={\date},
  ]
\placeTitle

\SlideTitle {Сетевые сокеты}
\starttyping
int sock_fd;
struct sockaddr_in addr;
unsigned char txbuf[100], rxbuf[100];

/* Инициализация addr и buf не показана. */

sock_fd = socket(AF_INET, SOCK_STREAM, 0);
connect(sock_fd, (struct sockaddr *)&addr, sizeof(addr));

write(sock_fd, txbuf, sizeof(txbuf));
read(sock_fd, rxbuf, sizeof(rxbuf));

send(sock_fd, txbuf, sizeof(txbuf), 0);
recv(sock_fd, rxbuf, sizeof(rxbuf), 0);

close(sock_fd);
\stoptyping

\SlideTitle {Средства передачи данных между процессами}
\IncludePicture[horizontal][diagrams/sockets.pdf]

\SlideTitle {TCP vs UDP}
\IncludePicture[horizontal][diagrams/tcp-udp.pdf]

\SlideTitle {Пакетная и потоковая передача данных}
\startitemize
\item При пакетной передаче границы сообщений сохраняются.
\item UDP обеспечивает пакетную передачу.
\item При потоковой передаче границы сообщений не сохраняются. Возможно
объединение исходных сообщений в одно, разбиение исходного сообщения на
несколько, а также любая комбинация этих операций.
\item TCP обеспечивает потоковую передачу.
\item Зачем вообще менять границы сообщений? Для некоторых приложений границы
не существенны (например, для передачи файла). При этом использование более
крупных сообщений снижает нагрузку на процессоры, а более мелких сообщений
обеспечивает передачу через локальные сети с маленьким максимальным размером
кадра.
\item Потоковая передача данных требует установления соединения.
\stopitemize

\SlideTitle {Что мы хотим от сетевых сокетов?}
\startitemize
\item Поддержка нескольких видов адресации: IP-адрес + порт для связи между
процессами на разных компьютерах, строковые имена - между процессами на одном
компьютере.
\item Поддержка пакетной (с сохранением границ сообщений) и потоковой (без
сохранения границ сообщения) передачи данных.
\item Средства установки и разрыва соединения.
\item Средства передачи и приема данных.
\item Общий программный интерфейс для сетевого и не сетевого обмена.
\item Сокрытие деталей реализации протоколов транспортного уровня в ядре
операционной системы.
\stopitemize

\SlideTitle {Классификация сокетов}
\starttyping
int socket(int domain, int type, int protocol);
\stoptyping
\startitemize
\item domain - семейство адресов:

\startitemize
\item AF_INET - адресом сокета является пара (IP-адрес, порт),
\item AF_UNIX - адресом сокета является строковые имя.
\stopitemize

\item type - тип сокета:

\startitemize
\item SOCK_DGRAM - пакетный (дейтаграммный) сокет (сохраняет границы сообщений,
не гарантирует доставку, допускает передачу данных без установки соединения),
\item SOCK_STREAM - потоковый сокет (не сохраняет границы сообщений,
гарантирует доставку, передача данных только после установки соединения).
\stopitemize

\item protocol - сетевой протокол ({\tt IPPROTO_TCP}, {\tt IPPROTO_UDP}, 0 -
протокол по умолчанию).
\stopitemize

\SlideTitle {Программный интерфейс сокетов}
\starttyping
int socket(int domain, int type, int protocol);
int connect(int sockfd, const struct sockaddr *addr, socklen_t addrlen);
ssize_t send(int sockfd, const void *buf, size_t len, int flags);
ssize_t sendto(int sockfd, const void *buf, size_t len, int flags,
               const struct sockaddr *dest_addr, socklen_t addrlen);
ssize_t recv(int sockfd, void *buf, size_t len, int flags);
int close(int fd);

int bind(int sockfd, const struct sockaddr *addr, socklen_t addrlen);
\stoptyping

\SlideTitle {Клиентские и серверные приложения}
\startitemize
\item Пример: Web-сервер и Web-браузер.
\item Будем называть клиентом то приложение, которое устанавливает соединение,
передает запросы и принимает ответы.
\item Будем называть сервером то приложение, которое принимает соединения,
принимает запросы и передает ответы.
\item Сервер имеет известный публичный адрес, клиенту адрес сокета (не путать с
IP-адресом) назначается динамически.
\item Клиент, использующий пакетный сокет, вызывает функции {\tt sendto} и {\tt
recv}.
\item Клиент, использующий потоковый сокет, вызывает функции {\tt connect},
{\tt send} и {\tt recv}.
\item Для назначения клиентскому сокету динамического адреса используется вызов
{\tt bind} со специальными аргументами (см. инструкцию к задаче).
\stopitemize

\SlideTitle {Адресация Unix-сокетов}
\startitemize
\item Адрес Unix-сокета - строковое имя. Точнее - имя файла.
\item Виртуальный файл с таким именем существует, но данные в нем не хранятся.
Его назначение - идентификация сокета.
\item Адрес задается структурой типа {\tt struct sockaddr_un}:
\starttyping
struct sockaddr_un {
	sa_family_t sun_family;               /* AF_UNIX */
	char        sun_path[108];            /* pathname */
};
\stoptyping
\item Пример использования:
\starttyping
	struct sockaddr_un addr;
	addr.sun_family = AF_UNIX;
	snprintf(addr.sun_path, sizeof(addr.sun_path), "%s", "/run/server");

	connect(sock_fd, (struct sockaddr *)&addr, sizeof(addr));
\stoptyping
\item {\tt struct sockaddr} - абстрактный адрес. Конкретные адреса ({\tt struct
sockaddr_un} и др.) приводятся к этому типа при передаче в функцию.
\stopitemize

\SlideTitle {Прием и передача данных через Unix-сокеты}
\starttyping
ssize_t recv(int sockfd, void *buf, size_t len, int flags);
\stoptyping
\startitemize
\item Для пакетного сокета {\tt recv} всегда принимает целый пакет. Буфер
должен быть не меньше размера принимаемого пакета. Возвращаемое значение -
размер пакета или индикация ошибки (-1).
\item Для потокового сокета {\tt recv} читает все имеющиеся данные,
помещающиеся в данный буфер. Возвращаемое значение - количество прочитанный
байтов или индикация ошибки (-1). Код 0 означает, что соединение было закрыто.
\item {\tt recv} - блокирующая функция. Она завершится только тогда, когда
будут прочитаны данные. Если данных нет, то исполнение приложения блокируется
до их появления.
\stopitemize

\SlideTitle {Упражнения}
\startitemize
\item Написать программу, которая отправляет учебному серверу запрос на чтение
строки из ячейки  0, принимает ответ, содержащий эту строку, и выводит
полученное значение в терминал. Учебный сервер хранит массив ячеек и
обрабатывает клиентские запросы по адресу /tmp/sock_dgram_server. Клиентское
приложение может отправлять запросы на чтение и запись значений этих ячеек.
Протокол описан в инструкции к практическому заданию.
\stopitemize

\stopcomponent
