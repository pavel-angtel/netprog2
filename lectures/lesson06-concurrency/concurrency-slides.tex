
\startcomponent concurrency-slides
\environment slides-env

\setupTitle
  [ title={ПО сетевых устройств},
   author={Трещановский Павел Александрович, к.т.н.},
     date={\date},
  ]
\placeTitle

\SlideTitle {Web-сервер с поддержкой нескольких соединений}
\IncludePicture[horizontal][diagrams/web-server.pdf]

\SlideTitle {Обработка одного запроса}
\IncludePicture[horizontal][diagrams/request.pdf]

\SlideTitle {Решение на основе параллельной обработки запросов}
\externalfigure[diagrams/parallel-solution.pdf][width=\textwidth]
 
\startitemize
\item Потоки - процессы с общим адресным пространством (общими данными).
\item Создание с помощью {\tt pthread_create} (обертка вокруг {\tt fork}).
\item Преимущество - использование нескольких ядер.
\item На одном ядре потоки исполняются поочередно.
\stopitemize

\SlideTitle {Недостатки}
\startitemize
\item Возможный конфликт при доступе к общим данным:

\starttabulate[|lT|lT|lT|]
\NC Поток 1 \NC Поток 2 \NC Результат \NR
\NC array[0] = 'a'; \NC \NC {'a', '-'} \NR
\NC \NC array[0] = 'b'; \NC {'b', '-'} \NR
\NC \NC array[1] = 'b'; \NC {'b', 'b'} \NR
\NC array[1] = 'a'; \NC \NC {'b', 'a'}  - недопустимая комбинация\NR
\stoptabulate
\item Необходима защита данных с помощью семафоров или мьютексов:
\starttyping
semaphore_down();
array[0] = 'a'; array[1] = 'a';
semaphore_up();
\stoptyping
\item Многопоточные приложения тяжело разрабатывать и отлаживать.
\item Если большую часть времени поток проводит в ожидании, нет выигрыша от
использования нескольких ядер.
\stopitemize

\SlideTitle {Решение на основе одновременной (concurrent) обработки запросов}
\IncludePicture[horizontal][diagrams/concurrent-solution.pdf]
 
\SlideTitle {Системный вызов {\tt select}}
\starttyping
int select(int nfds, fd_set *readfds, fd_set *, fd_set *, struct timeval *timeout);
\stoptyping
\startitemize
\item {\tt select} выполняет подписку на события и ожидание.
\item {\tt readfds} - множество файловых дескрипторов (от файлов, сокетов и др.).
\item {\tt select} спит до тех пор, пока хотя бы один из дескрипторов не будет
\emph{готов к чтению}. Готовность к чтению означает, что последующий {\tt read}
(или {\tt recv}) вернет данные без ожидания.
\item После возвращения из {\tt select} множество {\tt readfds} содержит только
те дескрипторы, которые готовы к чтению. Код возврата - общее количество
дескрипторов, готовых к чтению.
\item Аргумент {\tt timeout} задает максимальное время ожидания готовности.
Если это время истекает, {\tt select} возвращает 0.
\item Если {\tt timeout} установлен в {\tt NULL}, {\tt select} ждет готовности
неограниченное время.
\stopitemize

\SlideTitle {Работа с множеством файловых дескрипторов {\tt fd_set}}
\startitemize
\item {\tt fd_set} - битовая маска.
\item Задание множества:
\starttabulate[|lT|lT|]
\NC fd_set fds; \NC \NR
\NC int fd1 = 2, fd2 = 5; \NC \NR
\NC FD_ZERO(&fds); \NC $...00000000_2$ \NR
\NC FD_SET(fd1, &fds); \NC $...00000100_2$ \NR
\NC FD_SET(fd2, &fds); \NC $...00100100_2$ \NR
\stoptabulate
\item Проверка принадлежности множеству:
\starttyping
int fd1 = 2;
if (FD_ISSET(fd1, &fds)) {
	/* Если принадлежит. */
} else {
	/* Если не принадлежит. */
}
\stoptyping
\item Аргумент {\tt nfds} должен быть равен {\tt (max(fd1, fd2, ..., fdn) + 1)}.
\stopitemize

\SlideTitle {Работа со структурой {\tt struct timeval}}
\startitemize
\item Определение
\starttyping
struct timeval {
	time_t      tv_sec;     /* seconds */
	suseconds_t tv_usec;    /* microseconds */
};
\stoptyping
\item Получение текущего времени:
\starttyping
struct timeval tv;
gettimeofday(&tv, NULL);
\stoptyping

\item Арифметика:
\starttyping
struct timeval tv1, tv2, newtv;
timeradd(&tv1, &tv2, &newtv); /* newtv = tv1 + tv2 */
timersub(&tv1, &tv2, &newtv); /* newtv = tv1 - tv2 */
if (timercmp(&tv1, &tv2, <) {
	/* tv1 < tv2 */
} else {
	/* tv1 >= tv2 */
}
\stoptyping
\stopitemize

\SlideTitle {Простой цикл обработки событий (event loop)}
\starttyping
int listen_fd, sock_fd;
int max_fd, ret;
listen_fd = socket(...);
sock_fd = accept(listen_fd,...);

while (1) {
	struct fd_set fds;
	struct timeval timeout = {.tv_sec = 1};

	FD_ZERO(&fds);
	FD_SET(listen_fd, &fds);
	FD_SET(sock_fd, &fds);
	max_fd = listen_fd > sock_fd ? listen_fd : sock_fd;
	ret = select(max_fd + 1, &fds, NULL, NULL, &timeout);
	if (ret == 0) {
		/* Обработка таймаута. */
	} else if (FD_ISSET(listen_fd, &fds)) {
		/* Прием нового соединения. */
	} else if (FD_ISSET(sock_fd, &fds)) {
		/* Обработка запроса. */
	}
}
\stoptyping

\SlideTitle {Обобщенный цикл обработки событий (event loop)}
\starttyping
while (1) {
	struct fd_set fds;
	struct timeval timeout;

	FD_ZERO(&fds);
	for (...) { /* Перебор всех сокетов */
		FD_SET(sock_fd, &fds);
		max_fd = sock_fd > max_fd ? sock_fd : max_fd;

		... /* Модицикация timeout */
	}
	ret = select(max_fd + 1, &fds, NULL, NULL, &timeout);
	if (ret == 0) {
		/* Обработка таймаута. */
	} else {
		for (...) { /* Перебор всех сокетов */
			if (FD_ISSET(sock_fd, &fds)) {
				/* Обработка нового соединения или запроса. */
			}
		}
	}
}
\stoptyping

\SlideTitle {Блокирующий и неблокирующие вызовы}

\startitemize
\item {\tt select} должен быть единственной функцией в приложении, которой
разрешено спать.
\item Готовность дескриптора к чтению гарантирует, что 1 последующий вызов {\tt
read} вернется без блокирования процесса. Что делать, если требуется несколько
вызовов (например, приняли сразу несколько пакетов)?
\item Флаг {\tt MSG_DONTWAIT} делает вызовы {\tt recv} и {\tt send}
неблокирующими.
\item Если данных нет, возвращается ошибка {\tt EAGAIN} без блокировки:
\starttyping
	while (1) {
		ret = recv(sock_fd, buf, len, MSG_DONTWAIT);
		if (ret < 0 && errno == EAGAIN) {
			/* Нет данных */
			break;
		} else if (ret < 0) {
			/* Другая ошибка */
			return -1;
		}
	}
\stoptyping
\stopitemize

\SlideTitle {Определение таймаута при наличии нескольких таймеров}
\externalfigure[diagrams/timeout-calculation.pdf][width=\textwidth]
$timeout = min(T_1 - Current, T_2 - Current, ..., T_n - Current)$

\SlideTitle {Обработка таймаута при наличии нескольких таймеров}
\IncludePicture[horizontal][diagrams/timeout-handling.pdf]

\SlideTitle {Замечания по таймерам}
\startitemize
\item {\tt select} спит в течение времени, не меньшего, чем {\tt timeout}. Но
может спать дольше!
\item После таймаута надо проверять все таймеры, а не только самый ранний.
\item Вычислять новое значение таймаута надо от текущего момента, а не от
времени срабатывания предыдущего таймера.
\stopitemize

\SlideTitle {Упражнения}
\startitemize
\item Разработать приложение копирующее данные с файлового дескриптора 0 в
файловый дескриптор 1. Ожидание новых данных должно быть реализовано с помощью
функции {\tt select}. Дескриптор 0 должен быть переведен в неблокирующий режим
помощью функции {\tt fcntl}. Пример для дескриптора {\tt fd}:
\starttyping
	include <fcntl.h>

	fcntl(fd, F_SETFL, fcntl(fd, F_GETFL) | O_NONBLOCK);
\stoptyping
\item Разработать приложение, реализующее 10 одновременно работающих таймеров.
Таймаут для каждого таймера - случайное число в диапазоне от 1 до 10 секунд.
При срабатывании таймера необходимо выводить сообщение в терминал с номером
этого таймера. Для получения случайного числа использовать функцию {\tt rand}.
\stopitemize

\stopcomponent
