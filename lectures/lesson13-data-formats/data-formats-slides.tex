
\startcomponent data-formats-slides
\environment slides-env

\setupcombinations[distance=1cm]

\setupTitle
  [ title={ПО сетевых устройств},
   author={Трещановский Павел Александрович, к.т.н.},
     date={\date},
  ]
\placeTitle

\SlideTitle {Управляемые объекты в устройстве}
\IncludePicture[horizontal][diagrams/managed-objects.pdf]

\SlideTitle {Программно-конфигурируемые сети}
Современные сети (в частности, сети IoT) - больше, чем совокупность
составляющих их устройств.
\blank
Программно-конфигурируемая сеть (SDN - Software-Defined Network) - концепция
построения и администрирования сети, основанная на автоматическом
централизованном управлении всеми ресурсами сети за счет представления этой
сети в виде единого программируемого объекта.
\blank
Контроллер сети - объект, отвечающий за управление сетью SDN.

\SlideTitle {О каком управлении идет речь?}
\startitemize
\item Конфигурирование - приведение поведения устройства в соответствие с
некоторым документом или структурой данных.
\item Мониторинг.
\startitemize
\item Чтение статуса - текущее состояние системы (наличие соединений, состояние
конечных автоматов и пр.).
\item Чтение статистики - счетчики переданных пакетов, ошибок, отвергнутых
соединений и пр.
\item События - автономные сообщения об изменении состояния системы.
\stopitemize
\stopitemize

\SlideTitle {Пример}
\startitemize
\item Конфигурирование - задание IP-адреса:
\starttyping
# ip addr add dev eth0 192.168.0.8/24
\stoptyping
\item Чтение статуса - наличие (state UP) или отсутствие соединения.
\starttyping
# ip link
2: eth0: <BROADCAST,MULTICAST,UP,LOWER_UP> mtu 1500 qdisc pfifo_fast master lxcbr0 state UP mode DEFAULT group default qlen 1000
    link/ether 88:d7:f6:7f:98:84 brd ff:ff:ff:ff:ff:ff
\stoptyping
\item Чтение статистики.
\starttyping
ip -s link show dev enp7s0
...
RX: bytes  packets  errors  dropped overrun mcast
60078028   90936    0       0       0       10607
\stoptyping
\stopitemize

\SlideTitle {Традиционный подход к управлению сетевыми устройствами}
\startitemize
\item Специальный \quote{Cisco-подобный} CLI.
\item Web-интерфейс.
\item CLI операционной системы (shell), внутренние конфигурационные файлы.
\stopitemize

Концепцию SDN в рамках традиционного подхода реализовать тяжело.
\startitemize
\item \quote{Хрупкие} API - вывод команд и набор аргументов не стандартизован,
отличия между производителями и версиями ПО.
\item Не предусмотрена автоматическая обработка исключительных ситуаций.
Например, если из 10 команд одна выполнилась неуспешно. В каком состоянии в
итоге окажется система?
\item Ограниченные возможности - автономная отправка событий не предусмотрена.
\stopitemize

\SlideTitle {Потенциальные применения концепции SDN}
\startitemize
\item Масштабная автоматизация телекоммуникационных сетей.
\item Динамическое распределение пропускной способности на основе параметров
радио-каналов, загруженности сети и др..
\item Автоматическое управление сетью ЦОДа при создании или удалении
виртуальных машин.
\item Внедрение технологий \quote{умного} дома (города, предприятия)
предполагает наличие программируемой сети.
\stopitemize

\SlideTitle {Система управления устройством}
\IncludePicture[horizontal][diagrams/management-system.pdf]

\SlideTitle {Контроллер сети}
\IncludePicture[horizontal][diagrams/network-controller.pdf]

\SlideTitle {Документ JSON}
\starttyping
{
	"netowrk" : {
		"interfaces" : {
			"eth0" : {
				"enable" : true,
				"address" : "192.168.0.5",
				"prefix-length" : "28",
				"traffic-types" : ["multicast"]
			},
			"eth1" : {
				"enable" : false,
				"address" : "192.168.1.10",
				"prefix-length" : 22,
				"traffic-types" : ["unicast", "multicast"],
				"mtu" : 1400,
				"state" : {
					"used-by-process" : []
				}
			}
		}
	}
}
\stoptyping

\SlideTitle {Терминология и типы данных}
\startitemize
\item Формат представления структурированных данных.
\item JSON - JavaScript Object Notation, подмножество языка JavaScript.
\item Скалярные значения: целые числа (без кавычек), строки (с кавычками),
булевы значения.
\item {\em Объект} - контейнер для произвольных элементов JSON, индексируемый
строкой. Примерно соответствует словарю {\tt struct avl_tree}.
\item {\em Массив} - упорядоченная последовательность произвольных элементов
JSON. Примерно соответствует списку {\tt struct list_head}.
\item Библиотека на C - jansson.
\stopitemize

\SlideTitle {Особенности}
\startitemize
\item JSON обеспечивает сериализацию и десериализацию структурированных данных
в сетевых протоколах, конфигурационных файлах и т.д.
\item Самоописывающий формат - информация о структуре документа хранится в
самом документе (\{\}, [] и пр.). Упрощает создание расширяемых протоколов с
возможностью избирательной совместимости в каждой частной реализации.
\item Свободная структура документа - определение множества допустимых
документов возлагается на приложение. Для сравнения - в SQL это множество
определяется {\em схемой} базы данных.
\item Человекочитаемый формат. Можно просматривать и модифицировать в текстовом
редакторе.
\stopitemize

\SlideTitle {Представление документа в виде дерева {\tt json_t}}
\IncludePicture[horizontal][diagrams/json-t-tree.pdf]

\SlideTitle {Чтение и вывод документа}
\starttyping
int main(int argc, char *argv[])
{
	json_t *jconf;
	json_error_t jerr;
	char *str;

	/* json_t *json_load_file(const char *path, size_t flags, json_error_t *err) */
	jconf = json_load_file("configuration.json", 0, &jerr);
	if (!jsonf)
		goto on_error;

	/* char *json_dumps(const json_t *json, size_t flags); */
	/* JSON_ENCODE_ANY - корневой элемент не обязательно является объектом,
	 * JSON_INDENT - форматированный вывод с заданной шириной отступов. */
	str = json_dumps(jconf, JSON_ENCODE_ANY | JSON_INDENT(3));
	if (!str)
		goto on_error;

	printf("Document:\n%s\n", str);
	free(str);
	json_decref(jconf);
}
\stoptyping

\SlideTitle {Подробная информация об ошибках}
\starttyping
typedef struct json_error_t {
	int line; /* Строка, на которой возникла ошибка. */
	int column; /* Столбец, на котором возникла ошибка. */
	char text[JSON_ERROR_TEXT_LENGTH]; /* Описание ошибки. */
	...
} json_error_t;

int main(int argc, char *argv[])
{
	json_t *jconf;
	json_error_t jerr;

	jconf = json_load_file("configuration.json", 0, &jerr);
	if (!jconf) {
		fprintf(stderr, "Failed to load configuration: "
				"%s (line %d, column %d)\n",
				jerr.text, jerr.line, jerr.column);
		goto on_error;
	}
	json_decref(jconf);
}
\stoptyping

\SlideTitle {Типы {\tt json_t}}
\starttyping
typedef enum {
	JSON_OBJECT, JSON_ARRAY, JSON_STRING, JSON_INTEGER, 
	JSON_TRUE, JSON_FALSE, ...
} json_type;

int main(int argc, char *argv[])
{
	json_t *jdata = ...;
	json_type type;
	type = json_typeof(jdata);

	if (json_is_object(jdata))
		/* Использование функций json_object_*. */
	else if (json_is_array(jdata))
		/* Использование функций json_array_*. */
	else if (json_is_string(jdata))
		/* Использование функций json_string_*. */
	else if (json_is_integer(jdata))
		/* Использование функций json_integer_*. */
	else if (json_is_boolean(jdata))
		;
}
\stoptyping

\SlideTitle {Целые числа и булевы значения}
\starttyping
int main(int argc, char *argv[])
{
	json_t *jint, json_t *jbool;

	jint = json_integer(6);
	/* json_integer выделяет память, поэтому может завершиться ошибкой. */
	if (!jint)
		goto on_error;
	printf("Integer value: %d\n", json_integer_value(jint));
	json_decref(jint);

	jbool = json_boolean(1);
	/* json_boolean не выделяет память. */

	printf("Boolean value: %d\n", json_boolean_value(jbool));
	/* json_decref ничего не делает для булевых значений. */
	json_decref(jbool);
}
\stoptyping

\SlideTitle {Строки}
\starttyping
int main(int argc, char *argv[])
{
	char origin_str[] = "ABCDEFG";
	const char *new_str;
	json_t *jstr;

	jstr = json_string(origin_str);
	/* json_string копирует переданную строку. */
	if (!jstr)
		goto on_error;

	/* Исходную строку можно произвольно менять. Это не повлияет на jstr. */
	memset(origin_str, 0, sizeof(origin_str));

	new_str = json_string_value(jstr);
	/* json_string_value _не_ копирует строку. json_decref пока делать нельзя. */

	printf("String value: %s\n", new_str);

	/* Теперь можно. */
	json_decref(jstr);
}
\stoptyping

\SlideTitle {Объекты}
\starttyping
int main(int argc, char *argv[])
{
	json_t *jobj, *jchild;
	int ret;

	/* Создание пустого объекта. */
	jobj = json_object();
	if (!jobj)
		goto on_error;

	jchild = json_boolean(0);
	ret = json_object_set(jobj, "some-key", jchild);
	/* Добавление элемента в объект требует выделения памяти. */
	if (ret < 0)
		goto on_error;
	
	jchild = json_object_get(jobj, "some-key");
	json_object_del(jobj, "some-key");

	json_decref(jobj);
}
\stoptyping

\SlideTitle {Замечания}
\startitemize
\item {\tt json_object_get()} возвращает {\tt NULL}, если указанный ключ
отсутствует в объекте.
\item {\tt json_object_set()}  заменяет существующий дочерний элемент с
указанным ключом и освобождает его. Для сравнения - {\tt avl_insert} возвращает
-1 при добавлении двух элементов с одинаковыми ключами.
\item {\tt json_object_del()} возвращает -1, если указанный ключ отсутствует в
объекте.
\item {\tt json_object_del()} автоматически вызывает {\tt json_decref()} для
удаляемого элемента.
\item {\tt json_decref} обычно (см. далее более точное правило) освобождает
объект со всеми дочерними элементами. В отличие от {\tt struct avl_tree}
перебор дочерних элементов не требуется.
\stopitemize

\SlideTitle {Массивы}
\starttyping
int main(int argc, char *argv[])
{
	json_t *jarr, *jchild;
	int ret;

	/* Создание пустого массива. */
	jarr = json_array();
	if (!jarr)
		goto on_error;

	jchild = json_boolean(0);
	ret = json_array_append(jarr, jchild);
	/* Добавление элемента в массив требует выделения памяти. */
	if (ret < 0)
		goto on_error;
	
	/* Извлечение элемента с индексом 0. */
	jchild = json_array_get(jarr, 0);
	/* Удаление элемента с индексом 0. */
	json_array_remove(jarr, 0);
	json_decref(jarr);
}
\stoptyping

\SlideTitle {Перебор всех элементов}
\starttyping
int main(int argc, char *argv[])
{
	json_t *jobj, *jarr, *jchild;
	const char *key;
	int index;

	json_object_foreach(jobj, key, jchild) {
		printf("Object child: key %s, type %d\n", key, json_typef(jchild));

		/* Удалять jchild нельзя!. */
	}

	json_array_foreach(jarr, index, jchild) {
		printf("Array child: index %d, type %d\n", index, json_typef(jchild));

		/* Удалять jchild нельзя!. */
	}
}
\stoptyping

\SlideTitle {Совместное использование документа}
\IncludePicture[horizontal][diagrams/shared-tree.pdf]

\SlideTitle {Использование счетчика ссылок}
\IncludePicture[horizontal][diagrams/shared-tree-incref.pdf]

\SlideTitle {Освобождение части документа}
\IncludePicture[horizontal][diagrams/shared-tree-decref.pdf]

\SlideTitle {Правила использования счетчика ссылок}
\startitemize
\item В каждом {\tt json_t} есть счетчик ссылок {\tt refcount}.
\item При создании элемента его счетчику присваивается значение 1.
\item Вызов {\tt json_decref()} уменьшает счетчик на 1.
\item Когда значение счетчика становится равным 0, элемент удаляется.
\item Когда значение счетчика объекта или массива становится равным 0,
производится уменьшение на 1 счетчиков всех дочерних элементов.
\item Явное инкрементирование счетчика производится функцией {\tt json_incref()}.
\item Некоторые функции jansson инкрементируют счетчики обрабатываемых
элементов.
\stopitemize

\SlideTitle {Добавление без инкрементирования {\tt refcount}}
\starttyping
int main(int argc, char *argv[])
{
	json_t *jobj, *jchild;
	int ret;

	jobj = json_object();

	jchild = json_integer(10);

	/* Функция не инкрементирует refcount. jchild после вызова принадлежит jobj. */
	json_object_set_new(jobj, "some-key", jchild);
	
	/* Удаление jobj вместе с jchild, так как refcount jchild равен 1. */
	json_decref(jobj);

	/* Так делать нельзя! Повторное освобождение приведет к разрушению памяти. */
	json_decref(jchild);
}
\stoptyping

\SlideTitle {Добавление с инкрементированием {\tt refcount}}
\starttyping
int main(int argc, char *argv[])
{
	json_t *jobj, *jchild;
	int ret;

	jobj = json_object();

	jchild = json_integer(10);

	/* Функция инкрементирует refcount. */
	json_object_set(jobj, "some-key1", jchild);
	
	/* Можно добавить еще раз с другим ключом. */
	json_object_set(jobj, "some-key2", jchild);

	json_decref(jobj);

	/* У нас по-прежнему осталась наша ссылка.
	 * Если не вызвать json_decref, произойдет утечка памяти. */
	json_decref(jchild);
}
\stoptyping

\SlideTitle {Что такое \quote{ссылка}?}
\startitemize
\item Ссылка существует лишь неявно. Нет специальной структуры, представляющей
ссылку в программе.
\item Ссылка - обязательство вызвать {\tt json_decref()} после завершения
использования некоторого дерева {\tt json_t}.
\item Функция {\em заимствует} (borrow) ссылку, если она увеличивает {\tt
refcount} на 1 и гарантирует вызов {\tt json_decref()} в будущем (возможно,
после завершения функции).
\item Функция {\em ворует} (steal) ссылку, если она гарантирует вызов {\tt
json_decref()} в будущем, но не увеличивает {\tt refcount}.
\item Нежелательно создавать функции, которые в зависимости от условий либо
воруют, либо заимствуют ссылку.
\stopitemize

\SlideTitle {Пример}
\starttyping
int main(int argc, char *argv[])
{
	json_t *jobj, *jchild;
	int ret;

	jobj = json_object();

	jchild = json_integer(10);

	/* Функция ворует ссылку. */
	ret = json_object_set_new(jobj, "some-key", jchild);
	if (ret < 0) {
		/* Мы не добавили jchild в объект. Надо ли его освобождать? Нет. */
		/* json_decref(jchild); */
		goto on_error;
	}
	
	json_decref(jobj);
}
\stoptyping

\SlideTitle {Извлечение вложенных элементов}
\IncludePicture[horizontal][diagrams/unpack.pdf]

\SlideTitle {Использование {\tt json_unpack_ex()}}
\starttyping
{
	int enable;
	const char *address = NULL; /* Обязательно инициализировать! */

	/* '?' означает, что при отсутствии ключа переданный указатель не меняется. */
	ret = json_unpack_ex(jconf, &jerr, 0,
			     "{s:{s:b s?s}}",
			     "interfaces",
			     "enable", &enable,
			     "address", &address);
	if (ret < 0) {
		fprintf(stderr, "Invalid interface config: %s\n", jerr.text);
		return -1;
	}
	/* Что делать, если ключ "address" не был найден? */
	if (!address)
		address = "192.168.0.1";
	else
		/* address указывает на строку внутри json_t.
		 * Ее нельзя менять или освобождать. */
}
\stoptyping

\SlideTitle {Генерация дерева {\tt json_t}}
\starttyping
{
	int enable = 1;
	const char *address = "192.168.0.2";
	json_t *jchild1, *jchild2;

	jchild1 = json_integer(3);
	jchild2 = json_integer(4);

	/* 'o' - ссылка воруется,
	 * 'O' - ссылка заимствуется. */
	jconf = json_pack("{s:{s:b s:s s:o s:O}}",
			  "interfaces",
			  "enable", enable, /* Теперь значение, а не ссылка. */
			  "address", address,
			  "key1", jchild1,
			  "key2", jchild2);
	if (!jconf) {
		json_decref(jchild2);
		return -1;
	}
}
\stoptyping

\SlideTitle {Документ XML}
\starttyping
<network>
	<interface>
		<name>eth0</name>
		<enable>true</enable>
		<address>192.168.0.5</address>
		<prefix-length>28</prefix-length>
		<traffic-type>multicast</traffic-type>
	</interface>
	<interface>
		<name>eth1</name>
		<enable>false</enable>
		<address>192.168.1.10</address>
		<prefix-length>22</prefix-length>
		<traffic-type>unicast</traffic-type>
		<traffic-type>multicast</traffic-type>
		<mtu>1400</mtu>
	</interface>
</network>
\stoptyping
\stopcomponent
