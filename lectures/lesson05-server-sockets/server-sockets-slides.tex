
\startcomponent server-sockets-slides
\environment slides-env

\setupTitle
  [ title={ПО сетевых устройств},
   author={Трещановский Павел Александрович, к.т.н.},
     date={\date},
  ]
\placeTitle

\SlideTitle {Клиенты и сервер}
\IncludePicture[horizontal][diagrams/multiple-clients.pdf]

\SlideTitle {Вопросы}
\startitemize
\item Как указать номер TCP-порта в серверном приложении?
\item Как выбирается номер TCP-порта в клиентском приложении?
\item Как серверное приложение определяет источник (от какого клиента)
принимаемых данных?
\stopitemize

\SlideTitle {Слушающие и передающие сокеты}
\startitemize
\item В серверном приложении сетевое взаимодействие организовано с помощью 2
типов сокетов - {\em слушающего} сокет и {\em передающих} сокетов.
\item Слушающий сокет отвечает за установку TCP-соединений со всеми клиентами.
\item Слушающий сокет {\em привязывается} к определенному адресу и порту.
\item Передающий сокет отвечает за передачу данных между 1 клиентом и сервером.
\item Передающий сокет создается из слушающего сокета.
\item Номер порта передающего сокета выделяется ядром динамически.
\stopitemize

\SlideTitle {Серверный сокет до привязки}
\IncludePicture[horizontal][diagrams/before-binding.pdf]

\SlideTitle {Привязка сокета, API}
\starttyping
int sock;
struct sockaddr_in addr;
int ret;

sock = socket(AF_INET, SOCK_STREAM, 0);

addr.sin_family = AF_INET;
addr.sin_port = htons(port);
addr.sin_addr = {INADDR_ANY}; /* Привязка ко всем системным IP-адресам. */

/* int bind(int socket, const struct sockaddr *address, socklen_t addr_len); */
ret = bind(sock, (struct sockaddr *)&addr, sizeof(addr));
if (ret < 0) {
	fprintf(stderr, "Failed to bind to server: %s\n", strerror(errno));
	goto on_error;
}
\stoptyping

\SlideTitle {Очередь входящих соединений}
\IncludePicture[horizontal][diagrams/ingress-queue.pdf]

\SlideTitle {Перевод сокета в слушающий режим, API}
\starttyping
int sock;
struct sockaddr_in addr;
int ret;

sock = socket(AF_INET, SOCK_STREAM, 0);

/* ... */
bind(sock, (struct sockaddr *)&addr, sizeof(addr));

/* int listen(int socket, int backlog); */
ret = listen(sock, 5);
if (ret < 0) {
	fprintf(stderr, "Failed to put socket into listening state: %s\n",
			strerror(errno));
	goto on_error;
}
\stoptyping

\SlideTitle {Работа сокета в слушающем режиме}
\IncludePicture[horizontal][diagrams/listening-socket.pdf]

\SlideTitle {Передающий сокет}
\IncludePicture[horizontal][diagrams/transmitting-socket-creation.pdf]

\SlideTitle {Создание передающего сокета, API}
\starttyping
int listen_sock, data_sock;
struct sockaddr_in addr;
int ret;

listen_sock = socket(AF_INET, SOCK_STREAM, 0);

/* ... */
bind(listen_sock, (struct sockaddr *)&addr, sizeof(addr));
listen(listen_sock, 5);

/* int accept(int socket, struct sockaddr *address, socklen_t *addr_len); */
data_sock = accept(listen_sock, NULL, NULL);
if (data_sock < 0) {
	fprintf(stderr, "Failed to accept data connection: %s\n",
			strerror(errno));
	goto on_error;
}

ret = recv(data_sock, buf, sizeof(buf), 0);
\stoptyping

\SlideTitle {Работа сервера при создании передающих сокетов}
\IncludePicture[horizontal][diagrams/transmitting-socket.pdf]

\SlideTitle {Разрыв соединения со стороны клиента}
\starttyping
	int listen_sock, data_sock;
	struct sockaddr_in addr;
	int ret;

wait_connection:
	data_sock = accept(listen_sock, NULL, NULL);
	if (data_sock < 0) {
		goto on_error;
	}

	while (1) {
		char rx_buf[100], tx_buf[100];

		ret = recv(data_sock, rx_buf, sizeof(rx_buf), 0);
		if (ret == 0) {
			printf("Client disconnected\n");
			goto wait_connection;
		}

		/* Обработка запроса. */

		send(data_sock, tx_buf, sizeof(tx_buf), 0);
	}
\stoptyping


\SlideTitle {2 цикла обработки}
\IncludePicture[horizontal][diagrams/stream-server-algorithm.pdf]

\SlideTitle {2 цикла обработки, пример}
\starttyping
int listen_sock, data_sock;
struct sockaddr_in addr;
int ret;

listen_sock = socket(AF_INET, SOCK_STREAM, 0);
bind(listen_sock, (struct sockaddr *)&addr, sizeof(addr));
listen(listen_sock, 5);

while (1) {
	data_sock = accept(listen_sock, NULL, NULL);

	while (1) {
		char rx_buf[100], tx_buf[100];

		ret = recv(data_sock, rx_buf, sizeof(rx_buf), 0);
		if (ret == 0)
			break;

		/* Обработка запроса. */

		send(data_sock, tx_buf, sizeof(tx_buf), 0);
	}
}
\stoptyping

\SlideTitle {Замечания}
\startitemize
\item Функции {\tt accept()} и {\tt recv()} являются блокирующими.
\item Обмен данным в каждый момент времени производится максимум с одним
клиентом. Соединения остальных ждут в очереди.
\stopitemize

\SlideTitle {Web-технологии}
\IncludePicture[horizontal][diagrams/web.pdf]

\SlideTitle {Hyper-text transfer protocol (HTTP)}
\startitemize
\item Протокол обеспечивает удаленный доступ к сетевым {\em ресурсам}. Пример:
чтение HTML-страницы.
\item Ресурс идентифицируется с помощью URL - Uniform Resource Locator. Пример:
http://google.com/index.html.
\item Взаимодействие между клиентом и сервером по принципу запрос-ответ.
\item К указанному ресурсу применяется {\em метод}: чтение, запись, изменение и
др.
\item HTTP использует TCP в качестве протокола транспортного уровня.
\item Каждая пара запрос-ответ передается в отдельном соединении TCP.
\stopitemize

\SlideTitle {Uniform Resource Locator (URL)}
\setupTABLE[row][first][align=center]
\setupTABLE[row][each][frame=off]
\bTABLE
\bTABLEbody
\bTR
\bTD Синтаксис \eTD
\bTD <схема>://<доменное имя>:<порт>/<путь> \eTD
\eTR
\bTR
\bTD Пример 1 \eTD
\bTD http://angtel.ru/directory1/resource3 \eTD
\eTR
\bTR
\bTD Пример 2 \eTD
\bTD https://192.168.0.8/directory1/resource3 \eTD
\eTR
\eTABLEbody
\eTABLE

\startitemize
\item Применение не ограничено протоколом HTTP. URL идентифицирует абстрактный
сетевые ресурсы: файлы, люди, телефоны, почтовые ящики и др.
\item Схема - задает протокол и определяет синтаксис остальной части URL.
https- HTTP поверх протокола TLS.
\item Доменные имена преобразуются в IP-адреса с помощью протокола DNS.
\item Порт - опциональный TCP или UDP порт.
\item Путь интерпретируется HTTP-сервером. Не обязательно путь к файлу в
файловой системе.
\stopitemize

\SlideTitle {Методы HTTP}
\startitemize
\item GET - чтение содержимого ресурса.
\item PUT - задание нового содержимого ресурса.
\item POST - передача данных ресурсу.
\item DELETE - удаление ресурса.
\item PATCH - частичное изменение ресурса.
\stopitemize

\SlideTitle {Тип содержимого, MIME}
Содержимое ресурса передается в {\em теле} запроса или ответа. В зависимости от
ресурса формат данных в теле может быть разным. Этот формат указывается в
сообщении в соответствии со стандартом MIME (Multipurpose Internet Mail
Extensions).
\blank
Синтаксис: <тип>/<подтип>
\blank
Распространенные типы: image, text, audio, application.
\blank
Примеры: text/html, image/jpeg, application/pdf, application/json.
\blank
Тип содержимого указывается в {\em заголовке} Content-Type в HTTP сообщении.

\SlideTitle {Доступ к страницам HTML}
\IncludePicture[horizontal][diagrams/http-files.pdf]

\SlideTitle {Доступ к ресурсам других приложений}
\IncludePicture[horizontal][diagrams/http-external.pdf]

\SlideTitle {Доступ к внутренним ресурсам сервера}
\IncludePicture[horizontal][diagrams/http-inernal.pdf]

\SlideTitle {Программируемая сеть}
\startitemize
\item Протокол HTTP можно использовать отдельно от технологий HTML, CSS,
JavaScript и др. Клиентом в таком случае является не браузер, а некоторое
неитерактивное управляющее приложение.
\item REST (Representational State Transfer) - подход к построению API системы
на основе сетевых ресурсов и протокола HTTP.
\item REST - возможная основа программируемой сети устройств IoT.
\stopitemize

Пример - протокол RESTCONF:
\startitemize
\item Внутренние объекты системы (сетевые интерфейсы, таблицы маршрутизации,
датчики и т.д.) представляются в виде сетевых ресурсов.
\item Задание конфигурации производится через применение методов PUT, PATCH,
DELETE к соответствующим ресурсам.
\item Чтение конфигурации и статуса производится через применение метода GET.
\stopitemize

\SlideTitle {Синтаксис HTTP-запроса}
\starttyping
<Имя метода> <Путь> HTTP/1.1 \r\n
<Имя заголовка>: <Значение> \r\n
...
<Имя заголовка>: <Значение> \r\n
\r\n
<Тело запроса>
\stoptyping
Пример:
\starttyping
PUT /directory/resource2 HTTP1.1 \r\n
Host: angtel.ru \r\n
User-Agent: firefox \r\n
Content-Type: text/html \r\n
\r\n
<p> sdgsrthsrt <\p>
\stoptyping

\SlideTitle {Синтаксис HTTP-ответа}
\starttyping
HTTP/1.1 <Код> <Сообщение> \r\n
<Имя заголовка>: <Значение> \r\n
...
<Имя заголовка>: <Значение> \r\n
\r\n
<Тело ответа>
\stoptyping
Пример:
\starttyping
HTTP/1.1 301 Moved Permanently\r\n
Location: http://www.google.com\r\n
\r\n
\stoptyping

\SlideTitle {picohttprequest, API}
\starttyping
const char *method_ptr, *path_ptr;
struct phr_header headers[100];
size_t method_len, path_len, num_headers;
int minor_version;
int headers_len;

num_headers = 100;
/* buf - буфер с запросом HTTP;
 * len - размер буфера;
 * method_ptr, path_ptr - указатели на метод и путь в HTTP-запросе,
 *                        задаются функцией phr_parse_request;
 * method_len, path_len - длина метода и пути в HTTP-запросе,
 *                        задаются функцией phr_parse_request;
 * headers - массив объектов, описывающих заголовки запроса,
 *           заполняются функцией;
 * num_headers - на входе определяет максимальное количество заголовков,
 *               на выходе - обнаруженное количество заголовков.
 */
headers_len = phr_parse_request(buf, len, &method_ptr, &method_len, &path_ptr, &path_len,
                                &minor_version, headers, &num_headers, 0);
if (headers_len <= 0)
	return -1;
\stoptyping

\SlideTitle {picohttprequest, API 2}
\starttyping
const char *method_ptr, *path_ptr;
struct phr_header headers[100];
size_t method_len, path_len, num_headers;
int minor_version;
int headers_len;

num_headers = 100;
headers_len = phr_parse_request(buf, len, &method_ptr, &method_len, &path_ptr, &path_len,
                                &minor_version, headers, &num_headers, 0);
if (headers_len <= 0)
	return -1;

/* После вызова функции path_ptr указывает на символ внутри буфера buf, на
первый символ пути. Этот путь не ограничен нулевым терминатором '\0', так как
phr_parse_request не меняет содержимое исходного буфера buf. Поэтому
использовать path_str как C-строку нельзя! Для получения C-строки с путем
необходимо скопировать путь в отдельный буфер. */
snprintf(path, PATH_SIZE, "%.*s", (int)path_len, path_ptr);

snprintf(body, BODY_SIZE, "%.*s", (int)(len - headers_len), buf + headers_len);
\stoptyping

\SlideTitle {Формирование ответа, API}

\starttyping
int status = 200;
char status_msg[] = "Status message";
char body[] = "Response body";

char response[1024];

snprintf(response, sizeof(response), "HTTP/1.1 %d %s\r\n"
                                     "Content-Type: text/plain\r\n"
                                     "\r\n"
                                     "%s",
                                     status, status_msg, body);
\stoptyping

\SlideTitle {Работа с HTTP в shell}
Отправка запроса GET для ресурса resource1 на сайте google.com и отображение
тела ответа:
\starttyping
$ curl google.com/resource1
\stoptyping

Отправка запроса GET и отображение запроса и ответа со всеми заголовками:
\starttyping
$ curl google.com/resource1 -v
\stoptyping

Отправка запроса PUT с заданным телом:
\starttyping
$ curl google.com/resource2 -X PUT --data "New resource content"
\stoptyping

\SlideTitle {Повторная привязка сокета после перезапуска приложения}
\starttyping
int listen_sock = -1;
int option = 1;
struct sockaddr_in addr = {
	.sin_family = AF_INET,
	.sin_port = htons(port),
	.sin_addr = {INADDR_ANY},
};
int ret;

listen_sock = socket(AF_INET, SOCK_STREAM, 0);

/* Установка опции SO_REUSEADDR позволяет создавать и привязывать сокет
 * сразу после закрытия предыдущего сокета на этом порте. Ускоряет перезапуск
 * программы при отладке.
 */
ret = setsockopt(listen_sock, SOL_SOCKET, SO_REUSEADDR, &option, sizeof(option));
if (ret < 0) {
	fprintf(stderr, "Failed to set socket option: %s\n", strerror(errno));
	goto on_error;
}
\stoptyping

\stopcomponent
